\chapter{Problema}

Da anni ormai si tenta di risolvere il problema del riconoscimento di testi contenuti in immagine. Il problema è noto come Optical Character Recognition (OCR) e consiste nel riconoscere i caratteri di un testo contenuto in un'immagine. Il problema è complesso e presenta una serie di insidie che non sono di immediata risoluzione. Nonostante ciò, allo stato dell'arte esistono diversi algoritmi che consentono di ottenere risultati soddisfacenti.
Quello che viene presentato in questo documento è un modello che mira a semplificare il problema a una sottoclasse di immagini, avendo il vantaggio di ottenere un algoritmo più leggero ed efficiente, a discapito della sua versatilità.

Capita spesso che le immagini da cui è utile estrarre il testo siano screenshot. L'algoritmo presentato si occupa di estrarre il testo contenuto in uno screenshot, indipendentemente dal font e dai colori utilizzati. In realtà, viene inizialmente affrontato il problema assumendo che il testo presenti una sola parola. Tramite l'uso di euristiche, si può estendere facilmente l'implementazione comprendendo frasi (purché non siano divise su più righe).

TODO: immagini

\chapter{Soluzione}

TODO: spostare sto capitolo?

La soluzione proposta è realizzata tramite una pipeline composta da due fasi principali:
\begin{itemize}
	\item \textbf{Fase 1}: Suddivisione della parola in caratteri
	\item \textbf{Fase 2}: Classificazione del singolo carattere
\end{itemize}

Per la prima fase vengono utilizzare euristiche e algoritmi di image processing per suddividere la parola in caratteri. Per la seconda fase viene utilizzato un modello di deep learning per classificare i singoli caratteri, che viene approfondito nei capitoli successivi.

TODO: mettere schema pipeline
