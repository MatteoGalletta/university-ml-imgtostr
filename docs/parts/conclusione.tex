\chapter{Conclusione}

Il progetto ha approcciato il problema dell'OCR con un approccio ibrido che combina tecniche di image processing e deep learning. La relazione mostra la pipeline utilizzata e le euristiche, gli algoritmi e i modelli che ne fanno parte. Infine, si mostrano i risultati ottenuti, evidenziando le valutazioni complessive dell'intera pipeline e le prestazioni del modello finale.

I risultati hanno evidenziato come l'approccio individuato non sia particolarmente robusto, ma abbia comunque raggiunto un buon livello di accuratezza. Il modello ha mostrato una buona capacità di generalizzazione, ma è stato limitato dalla qualità del dataset e dalla sua ambiguità intrinseca dovuta all'architettura proposta.

Ulteriori evolutive che mantengono la semplicità dell'approccio utilizzato potrebbero includere l'estensione del riconoscimento di paragrafi, sfruttando la separazione di righe attraverso una proiezione laterale dell'immagine. Inoltre, si potrebbe considerare una modifica nella fase di generazione del dataset, andando a generare immagini non ambigue che il modello potrebbe riconoscere con maggiore facilità.

Durante lo sviluppo del progetto, sono stati appresi una serie di concetti e tecniche che potranno sicuramente essere utili in futuro. Vedere la curva di accuracy crescere insieme alla loss è stato affascinante e inaspettato. Il processo di training nella sua interezza ha richiesto del tempo per essere completato ma ha consentito di ottenere una dimestichezza con gli strumenti utilizzati, PyTorch in particolare. Un altro aspetto interessante è stato quello di dover trovare gli iperparametri ottimali per il modello, che ha richiesto un certo grado di sperimentazione che nessuno di noi si aspettava, mostrandoci come la ricerca della combinazione ottimale sia un processo cruciale ma complesso e non sempre lineare.