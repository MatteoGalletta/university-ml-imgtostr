\chapter{Codice}

Il codice del progetto è suddiviso in quattro cartelle ciascuna delle quali si occupa di un aspetto specifico del flusso. Di seguito è riportata la struttura delle cartelle e dei file princiapali, per avere una panoramica dell'organizzazione del progetto. \\ 

\tikzstyle{every node}=[draw=black,thick,anchor=west]
\tikzstyle{selected}=[draw=red,fill=red!30]
\tikzstyle{optional}=[dashed,fill=gray!50]
\begin{tikzpicture}[%
  grow via three points={one child at (0.5,-0.68) and
  two children at (0.5,-0.71) and (0.5,-1.45)},
  edge from parent path={(\tikzparentnode.south) |- (\tikzchildnode.west)}]
  \node {root/}
    child { node {core/}
        child{ node {ImageToStringClassifier.py}}
        child{ node {ImageToStringPostprocessing.py}}
        child{ node {ImageToStringPreprocessing.py}}
    }
    child [missing] {}
    child [missing] {}
    child [missing] {}
    child { node {dataset}
        child { node {dataset.ipynb}}
    }
    child [missing] {}
    child { node {demo/}
        child { node {demo.py}}
    }
    child [missing] {}
    child { node {src/}
        child { node {ImageToStringNet.py}}
        child { node {ImageToStringNetDropout.py}}
        child { node {main.ipynb}}
    }
    child [missing] {}
    child [missing] {}
    child [missing] {};
\end{tikzpicture}

\section{Core}
La cartella \emph{core} contiene il codice delle tre classi dedicate a preprocessing, classificazione e postprocessing.

\subsection{PreProcessing}
La classe \texttt{ImageToStringPreprocessing} prepara l'immagine contenente testo per la fase successiva di classificazione, segmentando le lettere e normalizzandole in un formato uniforme. A partire da un'immagine in input, esegue operazioni come conversione in scala di grigi, binarizzazione e inversione del contrasto se necessario. 
Successivamente, rileva e raggruppa le componenti connesse per identificare le singole lettere, calcolando anche informazioni spaziali come distanze relative e disegnando le relative bounding box sull'immagine originale.
Ogni lettera viene poi ritagliata, ridimensionata proporzionalmente e centrata su un'immagine nera 28x28, rendendola pronta per le fasi successive. La classe inoltre fornisce metodi per accedere all'immagine segmentata, alle lettere preprocessate e alla loro visualizzazione.

\subsection{Classificazione}
A partire da un'immagine contenente una sequenza di caratteri, la classe \texttt{ImageToStringClassifier} gestisce l'intero processo di riconoscimento integrando preprocessing, postprocessing e \texttt{ImageToStringNetDropout} per la classificazione.

\subsection{PostProcessing}
La classe \texttt{ImageToStringPostprocessing} a partire dalla lista delle lettere classificate con relative informazioni spaziali, applica le euristiche discusse nei capitoli precedenti per decidere dove inserire spazi tra parole, basandosi sulle distanze orizzontali tra i caratteri. Inoltre, sfrutta la posizione verticale delle lettere rispetto ai bounding box generale per correggere l'uso errato delle maiuscole e minuscole in caratteri ambigui, confrontando ciascun carattere incerto con il primo considerato affidabile. Il rsultato è una sequenza di caratteri più coerente, utile per migliorare l'output finale del sistema di OCR.


