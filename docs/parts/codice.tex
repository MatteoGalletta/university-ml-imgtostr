\chapter{Codice}

Il codice del progetto è suddiviso in quattro cartelle ciascuna delle quali si occupa di un aspetto specifico del flusso. Di seguito è riportata la struttura delle cartelle e dei file princiapali, per avere una panoramica dell'organizzazione del progetto. \\ 

\tikzstyle{every node}=[draw=black,thick,anchor=west]
\tikzstyle{selected}=[draw=red,fill=red!30]
\tikzstyle{optional}=[dashed,fill=gray!50]
\begin{tikzpicture}[%
  grow via three points={one child at (0.5,-0.68) and
  two children at (0.5,-0.71) and (0.5,-1.45)},
  edge from parent path={(\tikzparentnode.south) |- (\tikzchildnode.west)}]
  \node {root/}
    child { node {core/}
        child{ node {ImageToStringClassifier.py}}
        child{ node {ImageToStringPostprocessing.py}}
        child{ node {ImageToStringPreprocessing.py}}
    }
    child [missing] {}
    child [missing] {}
    child [missing] {}
    child { node {dataset}
        child { node {dataset.ipynb}}
    }
    child [missing] {}
    child { node {demo/}
        child { node {demo.py}}
    }
    child [missing] {}
    child { node {src/}
        child { node {ImageToStringNet.py}}
        child { node {ImageToStringNetDropout.py}}
        child { node {main.ipynb}}
    }
    child [missing] {}
    child [missing] {}
    child [missing] {};
\end{tikzpicture}

\section{Core}
La cartella \emph{core} contiene il codice delle tre classi dedicate a PreProcessing, PostProcessing e Classificazione.
\subsection{PreProcessing}
La classe \texttt{ImageToStringPreprocessing} gestisce l'elaborazione preliminare dell'immagine di input per prepararla alla fase successiva di classificazione ed è composta dai seguenti metodi:

\subsubsection*{Costruttore: \texttt{\_\_init\_\_(self, image)}}
\begin{itemize}
    \item Ricevere l'immagine originale RGB.
    \item Crea due versioni preprocessate dell'immagine:
    \begin{itemize}
        \item \texttt{image\_binary}: immagine binarizzata tramite \\\texttt{\_preprocess(to\_binary=True)}.
        \item \texttt{image\_gray}: immagine in scala di grigi normalizzata tramite \texttt{\_preprocess(to\_binary=False)}.
    \end{itemize}
    \item Chiama \texttt{\_segment\_and\_analyze()} che individua le lettere segementandole e ne analizza le caratteristiche; salva l'immagine con bounding box e le informazioni su ogni lettera.
    \item Chiama \texttt{\_extract\_letters()} per creare la lista di immagini delle singole lettere ridimensionate a 28x28 pixel.
\end{itemize}
\subsection{PostProcessing}
\subsection{Classificazione}
